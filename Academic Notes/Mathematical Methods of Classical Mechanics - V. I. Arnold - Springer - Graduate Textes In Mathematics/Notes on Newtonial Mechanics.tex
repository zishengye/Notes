\documentclass[conference]{IEEEtran}
\IEEEoverridecommandlockouts
% The preceding line is only needed to identify funding in the first footnote. If that is unneeded, please comment it out.
\usepackage{cite}
\usepackage{amsmath,amssymb,amsfonts,amsthm}
\usepackage{algorithmic}
\usepackage{graphicx}
\usepackage{textcomp}
\usepackage{xcolor}
\usepackage{tikz}
\ifCLASSOPTIONcompsoc
\usepackage[caption=false,font=normalsize,labelfont=sf,textfont=sf]{subfig}
\else
\usepackage[caption=false,font=footnotesize]{subfig}
\fi

\newtheorem{theorem}{Theorem}[section]

\theoremstyle{definition}
\newtheorem{definition}{Definition}[section]

\theoremstyle{remark}
\newtheorem{exmp}{Example}
 
\newtheorem{corollary}{Corollary}

\begin{document}
    \title{Notes on Newtonian Mechanics}

    \author{\IEEEauthorblockN{Zisheng Ye}}

    \maketitle

    \section{The principles of relaticity and determinacy}

    \subsection{Space and time}
    In Newtonian mechanics, space is three-dimensional and euclidean, and time is one dimensional.

    \subsection{Galileo's principle of relativity}
    \begin{enumerate}
        \item All the laws of nature at all moments of time are the same in all inertial coordinate systems.
        
        \item All coordinates systems in uniform rectilinear motion with respect to an inertial one are themselves inertial.
    \end{enumerate}

    \subsection{Newton's principle of determinacy}
    The initial state of a mechanical system, the totality of position and velocities of its points at some moment of time, uniquely determines all its motion.

\end{document}