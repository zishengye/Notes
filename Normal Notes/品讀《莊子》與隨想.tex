\documentclass[12pt,a4paper]{article}
\usepackage{CJKutf8}
\usepackage[top=1.5cm,bottom=2cm,left=2cm,right=2cm]{geometry}
\usepackage{indentfirst}
\usepackage{etoolbox}
\usepackage{relsize}
\usepackage{sectsty}

\AtBeginEnvironment{quotation}{\CJKfamily{bkai}\small}

\begin{document}
\begin{CJK*}{UTF8}{bsmi}
    
    \title{品讀『莊子』與隨想}
    \author{葉子晟}
    \date{}
    \maketitle
    
    \small
    本文參考由陳鼓應註譯,中華書局出版的『莊子今註今譯』全書共分上中下三冊,依次包含『莊子』的內篇,外篇和紮篇。本文不是研究莊子的論文,而是通過閱讀莊子談一些個人的感受和想法。會通過挑選一部分具有代表性的篇目,進行摘錄分析與賞析。通過聯繫作者個人的閱讀和人生經歷進行展開。閱讀經典只是一個過程,而結果在於如何將經典內化,聯繫到現實生活中,以期給自己個人的生活以啟發和指導。閱讀經典通常是一個漫長的過程,在閱讀的過程中,個人的想法也會因為個人境遇的變化而產生一定的改變。所以在文章中遇到一定程度的前後不一致有可能是由於作者個人的在時間跨度上的思維轉變產生的。希望每一位讀著能夠諒解。

    \normalsize

    \section{逍遙遊}
    逍遙遊是『莊子』通篇第一章,主要講述一個人當看透功名利祿、權勢地位的束縛,而使得個人精神達到悠遊自在、無掛無礙的境界。第一部分一上來,莊子就描繪了一幅非常奇幻的場景,一個廣大無窮的世界:
    \begin{quotation}
        北冥有魚,其名為鯤。鯤之大,不知其幾千里也。化而為鳥,其名為鵬。鵬之背,不知其幾千里也;怒而飛,其翼若垂天之雲。是鳥也,海運則將徙於南冥。南冥者,天池也。
    \end{quotation}
    逍遙遊專註於描寫一個宏大的世界,將其認定為人類的至高境界。莊子認為,這種至高境界是成就一番大事業的基礎,而且這種大境界需要深蓄厚養之後才能發揮效用。正所謂
    \begin{quotation}
        且夫水之積也不厚,則其負大舟也無力……風之積也不厚,則其負大翼也無力。故九萬里,則風斯在下矣,而後乃今培風;背負青天而莫之夭閼者,而後乃今將圖南。
    \end{quotation}
    莊子用這個比喻來說明,想要圖大事就需要做足夠的積累。現實中,這些積累通常是多種多樣的,成長過程中的各種成功與失敗,生活的閱歷,為人處世的學習與積累,都可以作為人生過程中的積累。而去觀察一些,所謂的名人,他們大都擁有不同於大多人的別樣經歷,而這些不一般的經歷給予了他們不同於常人的境界。

    \rightline{二零一九年,于美國賓夕法尼亞州匹茲堡}

\end{CJK*}
\end{document}